\documentclass{article}
\usepackage[utf8]{inputenc}
\usepackage{amsmath}
\usepackage{amsfonts}
\usepackage{amsthm}
\usepackage{graphicx}

\numberwithin{table}{section}
\numberwithin{equation}{section}
\numberwithin{figure}{section}

\newtheorem{theorem}{Theorem}
\numberwithin{theorem}{section}

\newtheorem{corollary}{Corollary}[theorem]


\title{Hello World}
\author{alinezhad.f }
\date{May 2022}

\begin{document}

\maketitle

\section{Introduction}

Let's begin with a formula $e^{i\pi}+1=0$. But we can also do:

\begin{enumerate}

\item $$ \frac{1}{n} $$

\item\[ \sum_{n=1}^{\infty}2^{-n}=1 \]

\item$$\sum_{n=1}^{\infty}2^{-n}=1$$

\item$$ e = \lim_{n\to\infty} \left(1+\frac{1}{n}\right)^n = 
\lim_{n\to\infty} \frac{n}{\sqrt[n]{n!}}
$$

\item$$e=\sum_{n=0}^{\infty} \frac{1}{n!}$$

\item$$\dots$$

\end{enumerate}


\subsection{Integration}

$$
\int_0^\infty f(x)dx
$$

$$
\iiint f(x,y,z) dx dy dz
$$

\subsection{Linear Algebra}

$$
\Vec{v}\cdot\Vec{w}
$$

$$
\begin{bmatrix}
1 & 2 & 3\\
4 & 5 & 6\\
7 & 8 & 9\\
\end{bmatrix}
$$

$$
\int _0^1\int _0^1 x y (x+y)dydx
$$

$$
\left(
\begin{array}{cc}
 \sqrt{38} & \left\{x_1\to 5 \sqrt{\frac{2}{19}},x_2\to 3 \sqrt{\frac{2}{19}}\right\} \\
\end{array}
\right)
$$

\vskip 3cm

\includegraphics[scale=0.15]{figure1.jpg}

\textit{hello}.

hello.


\section{Formatting}

\begin{enumerate}
    \item As a \textbf{bold}
    \item \textit{italic}.
    \item \underline{underline}

\end{enumerate}

\section{A better method of defining formulas}

\begin{enumerate}
    \item \begin{equation}
    \label{limit}
        \lim_{0\to\infty}{1}\frac{1}{n!}
    \end{equation}
    
    \item 
    \begin{equation}
    \begin{split}
        \label{limit e}
        e &= \lim_{n\to\infty} \left(1+\frac{1}{n}\right)^n \\
        \text{we can also add text here!}&=\lim_{n\to\infty} \frac{n}{\sqrt[n]{n!}}
    \end{split}
    \end{equation}

\end{enumerate}


\begin{multline}
    \label{multline}
        e = \lim_{n\to\infty} \left(1+\frac{1}{n}\right)^n = \lim_{n\to\infty} \left(1+\frac{1}{n}\right)^n=\lim_{n\to\infty} \frac{n}{\sqrt[n]{n!}} \\
        =\lim_{n\to\infty} \frac{n}{\sqrt[n]{n!}}=\lim_{n\to\infty} \frac{n}{\sqrt[n]{n!}}=\lim_{n\to\infty} \frac{n}{\sqrt[n]{n!}}
\end{multline}

Equation \ref{limit} is cool.

Equation \ref{limit e} is aligned.

Equation \ref{multline} is multi line,

\section{tables and figures}

\begin{table}[h] 
    \centering
    \begin{tabular}{|r|c|}
    \hline
       1  &  2\\    
       \hline
       3000000000  &  4\\
    \hline
    \end{tabular}
    \caption{test table}
    \label{table1}
\end{table}

this is a reference to table \ref{table1}


\begin{figure}[]
    \centering
    \includegraphics[width=\textwidth]{figure1.jpg}
    \caption{figure 1}
    \label{figure1}
\end{figure}

This is a reference to Fig \ref{figure1}

\section{Theorems}

\begin{theorem}[Test Theorem]
\label{testtheorem}
This is to test theorems.
\begin{proof}
This is to test the proof.
\end{proof}
\end{theorem}

\begin{corollary}
This is a corollary to theory \ref{testtheorem}.
\label{testcorollary}
\end{corollary}


this is referring to the theorem \ref{testtheorem}.

this is referring to the corollary \ref{testcorollary}.

\section{functions}

We could use this:\\
the real numbers $\mathbb{R}$

instead we do this:\\
\newcommand{\R}{\mathbb{R}}
the real numbers $\R$

also,\\
\newcommand{\cv}[2]{\begin{bmatrix}
#1\\
#2\\
\end{bmatrix}}

\end{document}


